\documentclass[a4paper,12pt]{scrartcl}
\usepackage[warn]{mathtext}
\usepackage[T2A]{fontenc}
\usepackage[utf8]{inputenc}
\usepackage[english,russian]{babel}

\usepackage{amsfonts}
\usepackage{amsmath}
\usepackage{amsthm}

\theoremstyle{definition}
\newtheorem{definition}{Определение}

\usepackage{makeidx}
\makeindex

\begin{document}

\section{Квадратичные поля}

Введём необходимые определения. Пусть, $L, K$ — поля, причём $K \subset L$.

\begin{definition}
  Элемент $\alpha \in L$ называется \index{элемент!алгебраический | \textit}
  \emph{алгебраическим над $K$}, если:
  $$
  \exists (f \in K[x], f \ne 0) \colon f(\alpha)=0.
  $$
\end{definition}

Пусть $\alpha$ --- алгебраичен над $K$. Рассмотрим множество всех многочленов из
$K[x]$, корнем которых он является: $I = {f \in K[x]: f(\alpha)=0}$. Это
множество является идеалом в кольце многочленов $K[x]$. Этот идеал является
главным с образующей --- $m(x)$ --- \index{многочлен!минимальный}
\emph{минимальным многочленом для \alpha.} Этот минимальный многочлен $m(x)$ не
приводим и отличен от нуля (иначе идеал совпадал бы со всем кольцом многочленов
от одной переменной над $K$).

\begin{definition}
  Расширение $L$ поля $K$ называется \index{Алгебраическое расширение числового
    поля | \textit} \emph{алгебраическим}, если каждый элемент $L$ алгебраичен
  над $K$.
\end{definition}

Мы можем рассматривать расширение числового поля как векторное пространство над
самим этим полем. Предположим, что мы хотим расширить поле $K$, добавив в него
некоторый элемент $\alpha$, и, тем самым, получив расширение $L$. В таком
случае, каждый элемент $l$ расширения можно представить в виде: $l = a + b \cdot
\alpha \text{, где } a,b \in K$. Очевидно, что базисом $L$ будет $(1, \alpha)$,
а размерность такого векторного пространства будет 2.

\begin{definition}
  \index{поле!степень расширения| \textit} \emph{Степенью расширения $L$ над
    полем $K$} называется размерность $L$ как векторного пространства над $K$.
  $$
  (L\colon K)=\dim_KL=\dim L \text{, где } L = <1,\alpha>
  $$
\end{definition}

\begin{definition}
  \index{квадратичное поле | \textit} \emph{Квадратичным полем} называется любое
  расширение поля рациональных чисел $\mathbb{Q}$ степени 2.
\end{definition}
Пусть рациональное, свободное от квалратов число $d \ne 1$, тогда  многочлен $x^2-d=0$
неприводим над полем $\mathbb{Q}$ и поле $\mathbb{Q}(\sqrt{d})$, полученное из
$\mathbb{Q}$ присоединением корня этого многочлена, является квадратичным.




\printindex

\end{document}